\documentclass[12pt]{amsart}
%%\usepackage{amsmath,amsfonts,amsthm} % loaded by amsart already
%\usepackage{mathtools,amssymb,multicol,parskip,mathabx} % uncomment as needed
%\usepackage[all]{xy}
\usepackage{graphicx}

%\usepackage{bbold}

\usepackage{hyperref}
\hypersetup{
	colorlinks,
	citecolor=black,
	filecolor=black,
	linkcolor=black,
	urlcolor=black
}

%\usepackage{tocloft}

%\usepackage{tikz} % Not needed as included in all modern TeX distros
%\usetikzlibrary{shapes,arrows}
\usepackage{tikz-cd} 
%\usepackage{hyperref}           % click to jump to refernce in .dvi
\usepackage{showkeys}           % shows the internal names of labels and references
\usepackage{ifsym}              % lots of symbols, weather, clocks, legends, random
\usepackage{calc}               % allows arithmetic calculation with quantities
\usepackage{enumerate}          % improved enumeration environment control
\usepackage{mathrsfs}             % Adds Script Fonts for Curly scripts see \scr below
\usepackage{pxfonts}            % Palatino text with pxfonts math -Nice
\usepackage{verbatim}           % verbatim classes
\usepackage{fancyhdr}           % super headers --recommended
\usepackage{url}                % appropriately display url's
\usepackage{gloss}              % some nice formatting specials
\usepackage[marginratio=1:1,height=8.5in,width=6.5in,tmargin=1.25in]{geometry}
			% 
%Margin Control
%\usepackage{yhmath}             % yh math fonts
\usepackage{comment}            % \begin{comment} environment, etc...
\usepackage[normalem]{ulem}     % allows for strikouts \sout{} etc...
%\usepackage[usenames,dvipsnames]{color}
\usepackage{todonotes}          % Allows for cool pointed margin-notes.


\setcounter{tocdepth}{3}

\makeatletter
\def\@tocline#1#2#3#4#5#6#7{\relax
  \ifnum #1>\c@tocdepth % then omit
  \else
    \par \addpenalty\@secpenalty\addvspace{#2}%
    \begingroup \hyphenpenalty\@M
    \@ifempty{#4}{%
      \@tempdima\csname r@tocindent\number#1\endcsname\relax
    }{%
      \@tempdima#4\relax
    }%%
    \parindent\z@ \leftskip#3\relax \advance\leftskip\@tempdima\relax
    \rightskip\@pnumwidth plus4em \parfillskip-\@pnumwidth
    #5\leavevmode\hskip-\@tempdima
      \ifcase #1
       \or\or \hskip 1em \or \hskip 2em \else \hskip 3em \fi%
      #6\nobreak\relax
    \dotfill\hbox to\@pnumwidth{\@tocpagenum{#7}}\par
    \nobreak
    \endgroup
  \fi}
\makeatother


\newcommand{\mb}{\mathbb}
\newcommand{\mc}{\mathcal}
\newcommand{\scr}{\mathscr}
\newcommand{\del}{\partial}
\DeclareMathOperator{\dev}{dev}
\newcommand{\la}{\lambda}
\newcommand{\eps}{\epsilon}
\newcommand{\ga}{\gamma}
\newcommand{\Ga}{\Gamma}


\newcommand{\veps}{\varepsilon}
\renewcommand{\d}{\partial}
\newcommand{\tth}{\textturnh}
\newcommand{\on}{\operatorname}
\newcommand{\tr}{\operatorname{tr}}
\newcommand{\hess}{\operatorname{Hess}}
\newcommand{\HH}{\mb{H}}
\newcommand{\RH}{\mb{H}}
\newcommand{\R}{\mb{R}}
\newcommand{\Z}{\mb{Z}}
\newcommand{\C}{\mb{C}}
\newcommand{\D}{\mb{D}}
\renewcommand{\P}{\mb{P}}
\newcommand{\E}{\mb{E}}
\renewcommand{\S}{\mb{S}}
\renewcommand{\Re}{Re}
\renewcommand{\Im}{Im}


\newcommand{\ol}{\overline}
\newcommand{\til}{\widetilde}
\newcommand{\wt}{\widetilde}
\newcommand{\op}[1]{\operatorname{#1}}
\newcommand{\set}[1]{\left\{#1 \right\}}
\newcommand{\ceil}[1]{\left\lceil #1 \right\rceil}
\newcommand{\bd}{\partial_\infty}
\newcommand{\of}{\circ }
\providecommand{\to}{\longrightarrow }
\newcommand{\into}{\hookrightarrow}
\newcommand{\abs}[1]{\left\lvert #1 \right\rvert }
\newcommand{\norm}[1]{\left\| #1 \right\| }
\newcommand{\inner}[1]{\left\langle #1 \right\rangle }
\newcommand{\conv}{\circledast}
\newcommand{\ds}{\displaystyle}
\newcommand{\bs}{\backslash}
\newcommand{\pa}{\partial}
\newcommand{\grad}{\nabla}
\DeclareMathOperator{\Vol}{Vol}
\renewcommand{\bar}{\overline}

\newtheorem{thm}{Theorem}[section]
\newtheorem{example}[thm]{Example}
\newtheorem{definition}[thm]{Definition}
\newtheorem{remark}[thm]{Remark}

\newtheorem{theorem}[thm]{Theorem}
\newtheorem*{theorem*}{Theorem}
\newtheorem{proposition}[thm]{Proposition}
\newtheorem{lemma}[thm]{Lemma}
\newtheorem{corollary}[thm]{Corollary}

\newcommand{\marginlabel}[1]
{\mbox{}\marginpar{\raggedleft\hspace{0pt}#1}}
\newcommand{\marg}[1]
{\mbox{}\marginpar{\tiny\hspace{0pt}#1}}
\newcommand{\numeq}[1]{\begin{align}\begin{split} #1
		\end{split}\end{align}}

\begin{document}

\author[Neal Coleman]{Neal Coleman$^\dagger$}
\thanks{$\dagger$ :)}

\title[Piecewise linear finite elements]{Finite elements}

\address{Chicagoland} %one \address command per author
\email{coleman.neal@gmail.com} % one \email command per author
%\curraddr{}
%\urladdr{} % use \textasciitilde instead of ~ in URL
%\dedicatory{}
%\date{\today} % not standard to put at bottom
%\translator{}
%\keywords{}
\subjclass[2000]{Primary 51R10; Secondary 51R12}

\begin{abstract}
Study the
\end{abstract}
\maketitle

%\markboth{Short author names}{Short title} % used for subsequent
%pages. less desirable alternative

\thispagestyle{empty} % turn off page numbering on title page

\section*{Introduction}



\section*{Triangle}

Let $p, q, r\in\mathbb{R}^2$ be the vertices of a triangle $T\subset\mathbb{R}^2$ with area $A_T$. Let $\theta_p$ be the interior angle at vertex $p$, and likewise $\theta_q, \theta_r$. Let $P$ be the length of the side opposite $p$, likewise with $Q$ and $R$. (We may sometimes abuse notation and denote by $P,Q,R$ the sides themselves.)

Area is computed by
\begin{align*}
A_T &= \frac{1}{2}\det\left[\begin{array}{c|c} p - r & q - r \end{array}\right] \\
	&= \frac{1}{2}\det\left[\begin{array}{c|c} p - q & r - q \end{array}\right] \\
	&= \frac{1}{2}\det\left[\begin{array}{c|c} q - p & r - p \end{array}\right] \\
\end{align*}

\section*{Barycentric coordinates}

Let $s = \begin{bmatrix}x\\y\end{bmatrix}\in\mathbb{R}^2$. Then we can write $s = \alpha_p p + \alpha_q q + \alpha_r r$, with $\alpha_p + \alpha_q + \alpha_r = 1$. Define
\[ B = \left[\begin{array}{c|c|c}
p & q & r \\
1 & 1 & 1 \\
\end{array}\right] \]
and
\[\alpha = \left[\begin{array}{c}
\alpha_p \\ \alpha_q \\ \alpha_r
\end{array}\right]\]
so that $x = B\alpha$.

Observe that
\begin{align*}
\det B &= \det\left[\begin{array}{c|c} q & r \end{array}\right]
	- \det\left[\begin{array}{c|c} p & r \end{array}\right]
	+ \det\left[\begin{array}{c|c} p & q \end{array}\right] \\
	&= (q_xr_y - q_yr_x) - (p_xr_y - p_yr_x) + (p_xq_y - p_yq_x) \\
	&= p_xq_y - p_xr_y - r_xq_y - p_yq_x + p_yr_x + r_yq_x - r_xr_y + r_xr_y \\
	&= (p_x - r_x)(q_y - r_y) - (p_y - r_y)(q_x - r_x) \\
	&= 2A_T
\end{align*}

Using cofactor expansion we calculate

\[B^{-1} = \frac{1}{2A_T}\left[\begin{array}{ccc}
q_y - r_y & -(q_x - r_x) & \det\left[\begin{array}{c|c} q & r \end{array}\right] \\
-(p_y - r_y) & p_x - r_x & -\det\left[\begin{array}{c|c} p & r \end{array}\right] \\
p_y - q_y & -(p_x - q_x) & \det\left[\begin{array}{c|c} p & q \end{array}\right] \\
\end{array}\right]\]

\section*{Building blocks of elements}

Define $v_p:T\to \mathbb{R}$ to be the restriction of the unique affine-linear function with $v_p(p) = 1$, $v_p(q) = 0$, $v_p(r) = 0$. Define $v_q$ and $v_r$ similarly.

In fact $v_p(s)$ is the $p$ coordinate in the barycentric expansion of $s = (x,y)$. (This is because linear functions are uniquely determined by their values.) If $e_i$ is the $i^{th}$ standard basis element of $\mathbb{R}^3$ we have
\[ v_p(s) = e_1^T B^{-1} s = \frac{1}{2A_T}\bigg( (q_y - r_y)x - (q_x - r_x)y + \det\left[\begin{array}{c|c} q & r \end{array}\right] \bigg)\]
and its gradient is constant and equal to
\[ \nabla v_p = \frac{1}{2A_T} \begin{bmatrix} q_y - r_y \\ -(q_x - r_x) \end{bmatrix} \]

\subsection*{Integrals}

We now compute $L^2$ norm of $v_p$ and the $L^2$ inner product of $v_p, v_q$.

To compute the squared norm $\|v_p\|^2$ of $v_p$:
\[ \|v_p\|^2 = \int_T (v_p)^2 = \int_0^1 t^2 |v_p^{-1}(t)|\ dt \]
Observing that $|v_p^{-1}(t)| = (1-t)P$ we have
\begin{align*}
\int_T (v_p)^2 &= \int_0^1 t^2(1-t)P\ dt\\
	& = P\bigg(\frac{1}{3} - \frac{1}{4}\bigg) \\
	& = \frac{1}{12}P
\end{align*}
Likewise $\|v_q\|^2 = \frac{1}{12}Q$ and $\|v_r\|^2 \frac{1}{12}R.$

To compute the inner product $\langle v_q, v_r\rangle_2$ of $v_p$ and $v_q$ we foliate $T$ by lines parallel to $P$, which are level sets of $v_p$.
\[ \langle v_q, v_r\rangle = \int_0^1 \int_{v_p^{-1}(t)} v_qv_r\ dt \]
We use a linear parametrization of $v_p^{-1}(t)$ with independent variable $s$, traversing from $tp + (1-t)q$ when $s=0$ to $tp + (1-t)r$ when $s=1$:
\[ s\mapsto s(tp + (1-t)r) + (1-s)(tp + (1-t)q) = tp + s(1-t)r + (1-s)(1-t)q\]
As the coefficients sum to one, they are barycentric coordinates. 
In abuse of notation re-use the symbols $v_q, v_r$ for their restrictions to $v_p^{-1}(t)$.  Then $v_q(s) = (1-s)(1-t)$ and $v_r(s) = s(1-t)$.

Thus
\[ \int_{v_p^{-1}(t)} v_qv_r = \int_0^1 s(1-s)(1-t)^2\ ds = \frac{1}{6}(1-t)^2 \]
and so
\begin{align*}
\langle v_q, v_r\rangle_2 &= \int_0^1 \frac{1}{6}(1-t)^2\ dt \\
	&= \frac{1}{18}
\end{align*}

The $L^2$ norm of $\nabla v_p$ is nothing more than
\[\|\nabla v_p \|^2 = \int_T\frac{(q_y - r_y)^2 + (q_x - r_x)^2}{4A_T^2} = \frac{P^2}{4A_T} \]

The inner product of $\nabla v_p$ and $\nabla v_q$ is
\[ \langle \nabla v_p, \nabla v_q \rangle = -\frac{1}{4A_T^2}\bigg( (q_y - r_y)(p_y - r_y) + (q_x - r_x)(p_x - r_x)\bigg) = -\frac{\langle q-r, p-r\rangle}{4A_T^2} \]
This can also be interpreted as proportional to the cosine of the interior angle at $r$:
\[ \langle \nabla v_p, \nabla v_q\rangle = -\frac{1}{4A_T}PQ\cos\theta_r\]

\section*{Triangulations and elements}

Let $\Omega\subset\mathbb{R}^2$ be a precompact domain with piecewise linear boundary. Let $\mathcal{T}$ be a triangulation of $\Omega$ such that:
\begin{itemize}
\item Each triangle is nondegenerate
\item No triangle has a vertex in the interior of its face
\item $\ldots$
\end{itemize}
Such a triangulation is ``regular.''

For a vertex $p$ of the triangulation, define
\[
\delta_p(x) = \begin{cases}
v_p(x), & x \mbox{ in the interior of a triangle adjacent to $p$ } \\
0, & \mbox{else}
\end{cases}
\]
It is easily checked that $\delta_p$ is piecewise linear.

The space $V_\mathcal{T}$ spanned by the $\delta_p$ is a finite-dimensional subspace of $H^1(\Omega)$. Problems in $H^1(\Omega)$ can be weakly approximated by linear algebra in $V_\mathcal{T}$.

The $H^1$ norm of $\delta_p$ can be computed as follows:
\begin{align*}
\| \nabla \delta_p \|^2 &= \sum_{T\ni p} \| \nabla v_p \|^2 \\
	&= \sum_{T\ni p} \frac{1}{12}(\mbox{length of far side of $T$})
\end{align*}



\section*{Computations with triangles}

Triangles admit a self-similar triangulation.

\section*{Proof of Weyl's law}





%**************************************************************************
% Text ends.
%**************************************************************************

%\bibliographystyle{amsalpha}
%\bibliography{allbib}
\providecommand{\bysame}{\leavevmode\hbox to3em{\hrulefill}\thinspace}
\providecommand{\MR}{\relax\ifhmode\unskip\space\fi MR }
% \MRhref is called by the amsart/book/proc definition of \MR.
\providecommand{\MRhref}[2]{%
  \href{http://www.ams.org/mathscinet-getitem?mr=#1}{#2}
} 
\providecommand{\href}[2]{#2}
\begin{thebibliography}{Rob02}
\bibitem[Kai91]{Kaimanovich:91}
Vadim~A. Kaimanovich, \emph{Poisson boundaries of random walks on discrete
  solvable groups}, Probability measures on groups, X (Oberwolfach, 1990),
  Plenum, New York, 1991, pp.~205--238. \MR{MR1178986 (94m:60014)}

\bibitem[Pra75]{Prat75}
J.-J. Prat, \emph{\'{E}tude asymptotique et convergence angulaire du mouvement
  brownien sur une vari\'et\'e \`a courbure n\'egative}, C. R. Acad. Sci. Paris
  S\'er. A-B \textbf{280} (1975), no.~22, Aiii, A1539--A1542.

\bibitem[Sul83]{Sullivan83}
D.~Sullivan, \emph{The {D}irichlet problem at infinity for a negatively curved
  manifold}, no.~18, 723--732.

\end{thebibliography}
\end{document}